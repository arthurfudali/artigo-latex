%%%% fatec-article.tex, 2024/03/10

%% Classe de documento
\documentclass[
landscape,
  a4paper,%% Tamanho de papel: a4paper, letterpaper (^), etc.
  12pt,%% Tamanho de fonte: 10pt (^), 11pt, 12pt, etc.
  english,%% Idioma secundário (penúltimo) (>)
  brazilian,%% Idioma primário (último) (>)
]{article}

%% Pacotes utilizados
\usepackage[]{fatec-article}
\usepackage{setspace}

%% Processamento de entradas (itens) do índice remissivo (makeindex)
%\makeindex%

%% Arquivo(s) de referências
%\addbibresource{fatec-article.bib}

%% Início do documento
\begin{document}

% Seções e subseções
%\section{Título de Seção Primária}%

%\subsection{Título de Seção Secundária}%

%\subsubsection{Título de Seção Terciária}%

%\paragraph{Título de seção quaternária}%

%\subparagraph{Título de seção quinária}%

%\section*{Diário de Bordo}%
\section*{Instruções para o preenchimento}
\doublespacing
\begin{enumerate}
    \item O Diário de Bordo é usado para registrar atividades, progressos, ideias e desafios enfrentados em um projeto ou durante a rotina de trabalho. Serve como um registro cronológico e detalhado das operações diárias, facilitando a organização e o acompanhamento das tarefas.
    \doublespacing
    \item Durante o registro das atividades deve-se incluir detalhes como datas, horários, descrições de tarefas, nomes de participantes e observações relevantes.  Esta documentação contínua ajuda na avaliação do progresso de projetos ou atividades, permitindo ajustes e melhorias contínuas nos processos.
    \doublespacing
    \item Para evidenciar a realização das tarefas, você poderá utilizar a criação de anexos para adicionar anotações, fotos, prints, questionários, entre outros.
\end{enumerate}

\break

\begin{table}[]
\begin{tabular}{|l|l|l|p{5cm}|p{8cm}|}
\hline
Nome da Atividade                     & Data de início  & Data de término & Responsável pela atividade  & Descrição da atividade realizada \\ \hline
Divisão de tarefas                    & 13/09/2024      & 13/09/2024      & Todos                       & Divisão das tarefas a serem feitas durante o projeto.\\ \hline
Planejamento do projeto               & 13/09/2024      & 13/09/2024      & Todos                       & Definição do nome do projeto e das funcionalidades.\\ \hline
Protótipo baixa fidelidade            & 13/09/2024      & 13/09/2024      & Amanda Oliveira             & Construção do protótipo inicial do sistema.\\ \hline
Diagrama de classes                   & 14/09/2024      & 25/10/2024      & Giovana Albanês             & \\ \hline
Protótipo média fidelidade            & 16/09/2024      & 26/09/2024      & Amanda Oliveira             & Criação das telas principais do sistema.\\ \hline
Criação do projeto em Java            & 17/09/2024      & 25/10/2024      & Arthur Fudali               & \\ \hline
Captura da webcam com JavaCV          & 17/09/2024      & 17/09/2024      & Arthur Fudali               & Criado o código de captura de imagem da webcam usando a biblioteca JavaCV. \href{https://www.youtube.com/watch?v=NUQc7-dYIxA&t}{Este foi o tutorial usado} \\ \hline
Reunião com o professor Luiz          & 17/09/2024      & 17/09/2024      & Todos                       & Explicamos o projeto e recebemos dicas de como prosseguir. \\ \hline
Estudos em Python                     & 18/09/2024      & 18/11/2024      & Giovana Albanês             & Início dos estudos da linguagem Python para entender como aplicar o Eye Tracking. \\ \hline
Aplicação Python                      & 18/09/2024      & 29/10/2024      & Giovana Albanês             & Início do desenvolvimento da aplicação Python de Eye-Tracking. \\ \hline
Detecção dos olhos com Python         & 21/09/2024      & 30/09/2024      & Giovana Albanês             & Detecção inicial dos olhos na webcam utilizando a linguagem Python. \\ \hline
Desenvolvimento da logo do projeto    & 21/09/2024      & 26/09/2024      & Amanda Oliveira             & Período de brainstorm e desenvolvimento da logo e identidade visual do projeto. \\ \hline
Protótipo de alta fidelidade          & 21/09/2024      & 02/11/2024      & Amanda Oliveira             & Desenvolvimento do protótipo de alta fidelidade do sistema. \\ \hline
Detecção dos movimentos da íris       & 30/09/2024      & 30/09/2024      & Giovana Albanês             & Detecção com sucesso do movimento da íris na webcam usando. \href{https://www.youtube.com/watch?v=WCb4OwmtEFU}{este tutorial.} \\ \hline
\end{tabular}
\end{table}

\break

\begin{table}[]
\begin{tabular}{|l|l|l|p{5cm}|p{8cm}|}
\hline
Nome da Atividade                     & Data de início  & Data de término & Responsável pela atividade  & Descrição da atividade realizada \\ \hline
Sessão de introdução do artigo        & 02/10/2024      & 08/10/2024      & Giovana Albanês             & Desenvolvimento da sessão de introdução do artigo científico. \\ \hline
Sessão de objetivos do artigo         & 02/10/2024      & 08/10/2024      & Amanda Oliveira             & Desenvolvimento da sessão de objetivos do artigo científico.\\ \hline
Pesquisa de artigos de referência     & 23/10/2024      & 05/11/2024      & Arthur Fudali               & Pesquisa de artigos de referência para o estado da arte. \\ \hline
Diagrama de objetos                   & 23/10/2024      & 25/10/2024      & Giovana Albanês             & Desenvolvimento do diagrama de objetos utilizando a ferramenta \href{https://www.lucidchart.com/pages/pt}{LucidChart.} \\ \hline
Código que cria Heatmaps v1.0         & 23/10/2024      & 23/10/2024      & Amanda Oliveira             & Desenvolvimento do código que cria um heatmap através do caminho do mouse com o Python. \\ \hline
Estudos sobre JavaCV e OpenCV         & 24/10/2024      & 25/10/2024      & Arthur Fudali               & Estudos focados na documentação das bibliotecas JavaCB e OpenCV para o protótipo em Java do Eye-Tracking. \\ \hline
Criação do protótipo em Java          & 24/10/2024      & 25/10/2024      & Arthur Fudali               & Feed da webcam já funcionando (desta vez com uma biblioteca a mais). \\ \hline
Modelo conceitual                     & 24/10/2024      & 31/10/2024      & Arthur Fudali e Giovana Albanês& Modelagem do modelo conceitual de Banco de Dados. \\ \hline
Modelo lógico                         & 24/10/2024      & 31/10/2024      & Arthur Fudali e Giovana Albanês& Modelagem do modelo lógico de Banco de Dados. \\ \hline
Junção de códigos                     & 25/10/2024      & 29/10/2024      & Giovana Albanês             & Implementação do código que cria heatmaps com o código de eye-tracking. \\ \hline
Reconhecimento do rosto e olhos       & 25/10/2024      & 25/10/2024      & Arthur Fudali               & Reconhecimento e marcação facil e ocular em tempo real baseado no input da webcam. \\ \hline
Roteiro para o Pitch                  & 25/10/2024      & 06/11/2024      & Amanda Oliveira             & Desenvolvimento do roteiro para o Pitch do projeto. \\ \hline
\end{tabular}
\end{table}

\break

\begin{table}[]
\begin{tabular}{|l|l|l|p{5cm}|p{8cm}|}
\hline
Nome da Atividade                     & Data de início  & Data de término & Responsável pela atividade  & Descrição da atividade realizada. \\ \hline
Sessão de estado da arte do artigo    & 25/10/2024      & 05/11/2024      & Arthur Fudali               & Desenvolvimento da sessão de estado da arte do artigo científico. \\ \hline
Sessão de resultados do artigo        & 25/10/2024      & 05/11/2024      & Giovana Albanês             & Desenvolvimento da sessão de resultados do artigo científico. \\ \hline
Modelagem do banco de dados físico    & 26/10/2024      & 29/10/2024      & Giovana Albanês             & Modelagem do banco de dados físico relacional. \\ \hline
Coleta de vídeos para o Pitch         & 26/10/2024      & 30/10/2024      & Amanda Oliveira             & Coleta de vídeos e geração de prompt usando a IA. \href{https://www.genmo.ai/}{Gemno}. \\ \hline
Criação e edição do Pitch             & 30/10/2024      & 06/11/2024      & Amanda Oliveira             & Criação e edição do Pitch usando a ferramenta. \href{https://filmora.wondershare.net}{Wondershare Filmora}. \\ \hline
Sessão de metodologia do artigo       & 04/11/2024      & 05/11/2024      & Arthur Fudali               & Desenvolvimento da sessão de metodologia do artigo científico. \\ \hline
Entrega da versão inicial do artigo   & 05/11/2024      & 05/11/2024      & Todos                       & Foi feito a entrega da versão inicial do artigo para ser corrigida pelo professor Luiz. \\ \hline
Design da landing page do grupo       & 05/11/2024      & 05/11/2024      & Amanda Oliveira             & Processo criativo da landing page inicial da equipe. \\ \hline
Codificação da landing page           & 05/11/2024      & 09/11/2024      & Amanda Oliveira             & Criação da landing page do grupo feito com HTML, CSS e JavaScript. \\ \hline
Caso de uso                           & 08/11/2024      & 12/11/2024      & Giovana Albanês             & Desenvolvimento do caso de uso. \\ \hline
Atualização do Pitch                  & 12/11/2024      & 12/11/2024      & Amanda Oliveira             & Atualização do Pitch conforme orientação dos professores (narração trocada, musica de fundo colocado e legenda em inglês). \\ \hline
Implementação do Merge Sort           & 13/11/2024      & 13/11/2024      & Arthur Fudali               & Implementação do algoritmo de ordenação Merge Sort no sistema. \\ \hline
Correção do artigo                    & 14/11/2024      & 14/11/2024      & Giovana Albanês             & Correção do artigo de acordo com as orientações passadas (ainda faltando corrigir o estado da arte). \\ \hline
\end{tabular}
\end{table}

\break

\begin{table}[]
\begin{tabular}{|l|l|l|p{5cm}|p{8cm}|}
\hline
Nome da Atividade                     & Data de início  & Data de término & Responsável pela atividade  & Descrição da atividade realizada. \\ \hline
Criação do banner                     & 15/11/2024      & 15/11/2024      & Amanda Oliveira             & Criação do banner para exposição na HubTEC'24. \\ \hline
Correção do estado da arte            & 15/11/2024      & 15/11/2024      & Arthur Fudali               & Correção das imagens do estado da arte de acordo com as orientações passadas. \\ \hline
Equação de recursão                   & 15/11/2024      & 15/11/2024      & Giovana Albanês             & Feita a equação de recursão que descreve o funcionamento do merge sort. \\ \hline
Diario de Bordo                       & 16/11/2024      & 18/11/2024      & Amanda Oliveira             & Passando a limpo as anotações do diario de bordo para o LaTeX. \\ \hline
\end{tabular}
\end{table}


\end{document}