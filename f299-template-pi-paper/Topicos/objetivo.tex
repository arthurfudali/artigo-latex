\subsection*{1. OBJETIVO GERAL}


Desenvolver um sistema inteligente de eye tracking que auxilie designers de experiência do usuário (UX) na otimização de interfaces gráficas, empregando técnicas de inteligência artificial e análise preditiva para gerar sugestões automáticas e mapas de calor, com base no comportamento visual dos usuários. O sistema visa antecipar ajustes necessários e aprimorar o processo de design.

\subsection*{2. OBJETIVOS ESPECIFICOS}
\begin{enumerate}
    \item 1. Implementar uma funcionalidade que permita o upload de designs, possibilitando que a inteligência artificial analise os dados e crie mapas de calor detalhados.
    \item 2. Desenvolver um modelo de inteligência artificial em Python, capaz de gerar sugestões de melhorias no design com base nas interações visuais dos usuários e aprendizado prévio.
    \item 3. Aplicar técnicas de visão computacional para capturar e analisar o comportamento visual dos usuários durante a interação com a interface.
    \item 4. Criar um módulo de testes com usuários reais, incluindo a captura do comportamento visual por meio de visão computacional durante o uso da interface.
    \item 5. Gerar mapas de calor que representem as áreas onde os usuários mantiveram a atenção por períodos prolongados.
    \item 6. Otimizar o trabalho do UX designer identificando áreas que precisam de ajustes com base na análise preditiva da IA, antes de um teste com usuários reais.
    \item 7. Validar a eficácia do sistema, assegurando que ele consiga antecipar e resolver pelo menos 70\% das alterações sugeridas pelos usuários finais, utilizando a análise preditiva da inteligência artificial.
\end{enumerate}






