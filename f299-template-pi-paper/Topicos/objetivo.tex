\subsection*{1. OBJETIVO GERAL}
Desenvolver um sistema inteligente de eye tracking que auxilie designers de experiência do usuário (UX) na otimização de interfaces gráficas. O sistema utilizará técnicas de visão computacional para capturar o comportamento visual dos usuários, gravando onde eles olham para formar mapas de calor que evidenciam as áreas de maior atenção e interação. O objetivo é antecipar ajustes necessários e aprimorar o processo de design, oferecendo dados precisos sobre a interação do usuário com a interface.

\subsection*{2. OBJETIVOS ESPECIFICOS}
\begin{enumerate}
    \item Aplicar técnicas de visão computacional para capturar e analisar o comportamento visual dos usuários durante a interação com a interface.
    \item Criar um módulo que permita a realização de testes com usuários reais, incluindo a captura do comportamento visual por meio de visão computacional durante o uso da interface.
    \item Criar algoritmos que transformem os dados de eye tracking em heatmaps, visualizando as áreas de maior atenção e interação dos usuários nos designs analisados.
    \item Realizar análises estatísticas e qualitativas dos dados coletados para entender como os usuários interagem com os elementos da interface, identificando padrões de comportamento.
    \item Implementar uma funcionalidade que permita o upload de designs, possibilitando que a inteligência artificial analise os dados.
    \item Desenvolver um modelo de inteligência artificial em Python, capaz de gerar avaliações sobre o design com base nas interações visuais dos usuários e aprendizado prévio.
    \item Otimizar o trabalho do UX designer, desenvolvendo um modelo de inteligência artificial capaz de gerar heatmaps nas telas enviadas.
    \item Implementar um módulo para visualizar as avaliações e heatmaps dos designs enviados para a inteligência artificial.
    \item Implementar um módulo que permita visualizar as avaliações e heatmaps dos designs analisados através de eye tracking.
    \item Criar gráficos que compilem e apresentem a distribuição das notas baixas atribuídas pela inteligência artificial em diferentes quesitos, oferecendo insights sobre áreas a serem melhoradas.
    \item Produzir gráficos que ilustrem a taxa de satisfação que a IA supõe que os clientes teriam em resposta a melhorias no design, variando de insatisfeito a satisfeito.
    \item Conduzir testes de usabilidade para validar a eficácia do sistema de eye tracking e avaliações por inteligência artificial, coletando feedback dos usuários para ajustes e melhorias.
    \item Documentar os resultados obtidos durante a pesquisa, incluindo o impacto da análise dos heatmaps e avaliações preditivas na experiência do usuário.
\end{enumerate}






