O projeto visa desenvolver uma ferramenta web de apoio aos designers de UX, que integra Eye Tracking, HeatMaps e Inteligência Artificial para auxiliar na validação de interfaces. O foco principal do sistema é o designer, permitindo ao cliente participar apenas da fase de rastreamento ocular, quando necessário.

O método proposto baseia-se na construção de um sistema web destinado aos designers de UX, oferecendo a possibilidade de consulta à IA para análise preditiva. Neste processo, o designer pode enviar uma tela para a IA, que retorna um heatmap e uma escala de satisfação de 1 a 5, considerando quatro critérios principais: usabilidade, atratividade visual, clareza das informações e fluxo da navegação.

Para que a IA possa oferecer essas análises, é necessário alimentá-la com dados derivados de uma matriz de rastreamento ocular, que armazena a frequência e a intensidade do foco visual durante os testes. A coleta desses dados ocorre quando o cliente interage com a interface sendo testada, enquanto o sistema de Eye Tracking executa em segundo plano. A matriz registra a intensidade do olhar em diferentes áreas da tela, indicando as regiões de maior e menor interesse.

Ao final do teste, esses valores são processados para gerar o heatmap da tela analisada, e os dados resultantes são utilizados para treinar a IA, criando um modelo preditivo para futuras consultas. Em uma primeira fase, os testes serão realizados com alunos da Fatec, e em um segundo momento, com clientes reais do designer, para um treinamento contínuo da IA.

Com o software e a IA treinados, espera-se oferecer uma ferramenta que permita ao designer consultar a IA para obter uma análise preditiva, sem a necessidade de repetir testes com usuários reais em todas as etapas. Isso visa otimizar o processo de design, mantendo a qualidade da interface com base em análises automatizadas e sustentadas pelos dados de rastreamento ocular.

Para desenvolver a landing page da equipe do projeto, foi utilizada a linguagem de marcação HyperText Markup Language (HTML), responsável por definir a estrutura e o conteúdo principal da página. A estilização visual foi realizada com Cascading Style Sheets (CSS), o que permitiu aplicar cores, espaçamentos, tipografias e outros elementos de design, assegurando que a página estivesse visualmente alinhada com a identidade da equipe. Além disso, utilizou-se JavaScript, que introduziu interatividade e dinamismo à página, proporcionando animações e ações responsivas, aprimorando a experiência de navegação e tornando o conteúdo mais atraente para o usuário.

Neste projeto, a linguagem Java foi escolhida para implementar o sistema de rastreamento ocular devido à sua portabilidade e facilidade de integração com interfaces gráficas. Utilizando a biblioteca OpenCV em conjunto com classificadores Haar, foi possível detectar o rosto e os olhos do usuário automaticamente, ao ativar a webcam do computador. Esse método permitiu uma identificação inicial eficiente das regiões faciais para o rastreamento ocular. Contudo, observou-se que a linguagem Java apresenta limitações em termos de bibliotecas e recursos especializados para análise avançada em tempo real, restringindo a robustez e o desempenho do sistema. Essa limitação ressalta a importância de avaliar linguagens alternativas, como Python, que oferece uma gama mais ampla de recursos e suporte para visão computacional e processamento de imagens.

No desenvolvimento do sistema de rastreamento ocular avançado (Eye Tracking), optou-se pela linguagem Python devido à sua compatibilidade com bibliotecas de visão computacional e análise de dados, bem como à sua eficiência e facilidade de aprendizado. Python permite a criação de programas com menos linhas de código, o que aumenta a produtividade dos desenvolvedores e agiliza o desenvolvimento. Além disso, sua grande biblioteca padrão contém diversos módulos reutilizáveis, que eliminam a necessidade de escrever código do zero para muitas tarefas. \textcite{Amazon}

Com o intuito de capturar imagens em tempo real da câmera, utilizou-se a biblioteca OpenCv, e o framework MediaPipe Face Mesh foi escolhido para a detecção facial. O Face Mesh do MediaPipe identifica 468 landmarks (pontos de referência) na face, mapeando características como olhos, boca, nariz e contorno facial. Esse mapeamento detalhado permite rastrear micro movimentos da íris, possibilitando identificar a direção e o foco visual em tempo real.

Além dessas ferramentas, a biblioteca NumPy foi utilizada para estruturar os dados de coordenadas em matrizes, e a biblioteca JSON auxiliou no armazenamento dos dados. A visualização do mapa de calor foi realizada com PyGame e Matplotlib, destacando as áreas de maior foco visual. Por fim, a biblioteca mysql.connector foi utilizada para integrar o sistema a um banco de dados em tempo real, armazenando os dados de rastreamento ocular para consultas e análises futuras.

Python ainda oferece a vantagem de portabilidade, podendo ser executado em diversos sistemas operacionais como Windows, macOS, Linux e Unix. Sua ampla comunidade global de suporte facilita o aprendizado e proporciona soluções rápidas a problemas, contribuindo para a manutenção e evolução contínua do sistema. \textcite{Amazon}

Para o desenvolvimento dos códigos mencionados, utilizou-se o Visual Studio Code (VS Code), um Ambiente de Desenvolvimento Integrado (IDE) criado pela Microsoft em 2015. O VS Code é um editor de código aberto amplamente reconhecido e utilizado na comunidade de desenvolvimento por sua versatilidade e eficiência. Ele suporta diversas linguagens de programação, oferecendo uma interface amigável e funcionalidades que potencializam o processo de desenvolvimento. \textcite{Akira}

Em relação a diagramação, foi utilizada a Unified Modeling Language (UML), desenvolvida por Grady Booch, James Rumbaugh e Ivar Jacobson, que serve para documentar projetos de software. A UML pode ser usada para visualizar, especificar, construir e documentar os artefatos de um sistema de software. No atual projeto, foram construídos dois diagramas UML, sendo eles o Diagrama de Classes e o Diagrama de Objetos.

O diagrama de classes é uma representação visual da estrutura de um sistema, descrevendo classes, seus atributos, métodos e os relacionamentos entre elas. Enquanto o Diagrama de Objetos é uma representação visual que exibe instâncias específicas de classes em um momento particular do sistema, mostrando objetos com seus valores atuais de atributos e as relações entre eles. Diferente do diagrama de classes, que é mais abstrato e define a estrutura geral, o diagrama de objetos detalha uma visão concreta do estado do sistema em determinado instante.

Para a diagramação dos modelos UML no projeto, foi utilizada a ferramenta LucidChart. O LucidChart é uma plataforma de criação de diagramas baseada na nuvem, amplamente usada para projetar e documentar sistemas complexos de forma visual.

A modelagem de banco de dados foi feita utilizando a ferramenta BrModelo, tanto o modelo conceitual, quanto o modelo lógico. A ferramenta brModelo foi desenvolvida pelo Grupo de Banco de Dados da UFSC em 2005 com o intuito de ser uma ferramenta gratuita para apoiar o ensino de projeto de bancos de dados relacionais. O modelo de banco de dados conceitual oferece uma visão abstrata das entidades e seus relacionamentos, focando nas necessidades de dados e suas interações. O modelo lógico traduz esses conceitos para uma estrutura que pode ser implementada tecnicamente. Juntos, esses modelos fornecem uma base sólida para o desenvolvimento do banco de dados da ferramenta web. \textcite{SBBD}

Em relação à elaboração do banco de dados relacional físico foi utilizada a linguagem de consulta estruturada (SQL). Por meio de seus comandos é possível armazenar, atualizar, remover, pesquisar e recuperar informações do banco de dados relacional, o qual é organizado em formato tabular, com linhas e colunas representando diferentes atributos de dados e as várias relações entre os valores dos dados. \textcite{Amazon}

Como ferramenta, foi utilizado o MySQL Workbench, uma solução visual de banco de dados projetada para o sistema gerenciador de banco de dados MySQL. A ferramenta facilita a criação e o gerenciamento de modelos de dados através de uma interface intuitiva, permitindo que diagramas de banco de dados sejam desenvolvidos de forma visual. Além disso, o MySQL Workbench proporciona uma interface robusta para escrever e executar consultas SQL, bem como para desenvolver stored procedures e funções, tornando o processo de desenvolvimento mais ágil e organizado. \textcite{DNC}

Para desenvolver a logo do Ueye, bem como sua identidade visual e, até mesmo a prototipação de suas telas, foi utilizada a plataforma colaborativa de design Figma. Essa ferramenta permite criar interfaces e protótipos de forma intuitiva, possibilitando a construção de fluxos de navegação e layouts de maneira integrada. Além disso, o Figma favorece a colaboração em tempo real entre designers e demais membros da equipe, o que contribuiu para uma criação mais eficiente e alinhada com os objetivos do projeto.  \textcite{Alura}

Além disso, utilizamos a ferramenta SEBRAE Canvas para estruturar o modelo de negócios do software. Um modelo de negócios descreve como uma organização cria, entrega e captura valor, abordando os principais componentes que influenciam o sucesso de uma empresa. Essa ferramenta oferece um quadro segmentado em áreas como parcerias, proposta de valor, estrutura de custos e fontes de receita, facilitando o planejamento de cada componente do modelo de negócios e a definição dos elementos essenciais do projeto. Vale ressaltar que o Canvas foi criado antes do desenvolvimento do aplicativo, pois ele proporciona uma visão clara e abrangente do sistema.

Para a organização e gestão de tarefas no desenvolvimento do Ueye, adotou-se uma metodologia ágil e uma ferramenta que suportam o trabalho colaborativo e a eficiência na entrega de valor. A metodologia utilizada é o Kanban, um sistema visual que ajuda a otimizar o fluxo de trabalho. Desenvolvido inicialmente no Japão pela Toyota, o Kanban permite gerenciar tarefas em colunas, como “A Fazer”, “Em Progresso” e “Concluído”, mantendo uma visão clara do progresso e das prioridades. \textcite{TOTVS}

Com o intuito de suportar a aplicação dessa metodologia, utilizamos o Trello, uma ferramenta de gerenciamento de projetos baseada em quadros visuais. Com o Trello, as equipes podem organizar tarefas em cartões que são movidos entre listas, permitindo visualizar o progresso e facilitar a colaboração. Cada cartão pode ser enriquecido com checklists, datas de vencimento e responsáveis, proporcionando uma visão detalhada de cada tarefa. Assim, o uso do Trello, combinado com a metodologia Kanban, otimizou a organização e execução das tarefas no desenvolvimento do software Ueye, garantindo eficiência e colaboração entre os membros da equipe. \textcite{Magalhães}