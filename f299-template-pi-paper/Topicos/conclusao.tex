Para auxiliar no cumprimento dos Objetivos de Desenvolvimento Sustentável (ODS), estabelecidos pela Organização das Nações Unidas (ONU), o projeto atual busca contribuir para o nono objetivo, que promove a Inovação e a Infraestrutura, ao desenvolver uma nova abordagem para a análise do design de interfaces de usuário. \textcite{ODS2024}

O sistema será desenvolvido para apoiar designers de UX, integrando Eye Tracking, Heatmaps e Inteligência Artificial, com o objetivo de otimizar o processo de validação de interfaces. A ferramenta permitirá que o designer envie telas para análise preditiva da IA, que retornará um heatmap e uma escala de satisfação baseados em critérios como usabilidade e fluxo de navegação. O sistema registra dados de rastreamento ocular durante os testes, gerando um modelo preditivo para análises futuras, eliminando a necessidade de repetição de testes com usuários reais. Assim, a ferramenta visa agilizar o processo de design, mantendo a qualidade da interface por meio de análises automatizadas.

Todo o processo será realizado por meio do software uEye, que integrará tanto o Eye Tracking quanto a consulta à Inteligência Artificial.

Até o momento, alcançamos a integração bem-sucedida de bibliotecas como OpenCV e MediaPipe para realizar a detecção facial e ocular, bem como a geração de mapas de calor utilizando PyGame e Matplotlib. O sistema está em funcionamento, coletando dados de interação dos usuários. O banco de dados relacional desenvolvido com MySQL está estruturado para armazenar informações sobre os testes de rastreamento ocular, possibilitando o treinamento futuro da IA por meio de consultas em tempo real.

Ao analisar os artefatos do projeto, percebe-se que o estudo atual ainda demanda testes práticos em contextos reais com profissionais especializados, a fim de obter relatórios e avaliações que possibilitem a melhoria contínua do sistema. Além disso, há a necessidade de realizar estudos mais intensivos sobre Testes de Usabilidade e Eye Tracking para aprimorar a precisão dos dados fornecidos, especialmente no que se refere ao rastreamento ocular. Esses testes também fornecerão um retorno valioso sobre a eficácia e usabilidade da análise preditiva de modelos de UX de alta fidelidade orientados pelo rastreamento ocular.

No futuro, o projeto prevê a inclusão de inteligência artificial para aprimorar os resultados. Inicialmente, planeja-se o treinamento de dois modelos de IA com objetivos distintos. O primeiro modelo será treinado utilizando as imagens dos mapas de calor gerados durante o processo de rastreamento ocular. Já o segundo modelo será alimentado com as telas de design avaliadas pelos clientes, juntamente com os feedbacks fornecidos. Esse feedback incluirá notas em uma escala de 1 a 5 para critérios como "Usabilidade", "Atratividade visual", "Clareza das informações" e "Fluxo de navegação".

Com isso, espera-se que a primeira IA seja capaz de prever possíveis mapas de calor para uma tela específica enviada pelo designer, eliminando a necessidade de novos testes de rastreamento ocular com clientes reais. Paralelamente, a segunda IA poderá replicar o feedback que seria dado por um usuário real, tornando o processo mais eficiente e automatizado.

No que se refere à identificação dos dados de forma intuitiva em mapas de calor, propõe-se o uso do processo de homogenização. Nesse processo, as áreas de foco visual do usuário receberão um valor estimado, criando uma representação contínua dos dados, suavizando as variações e permitindo uma visualização aprimorada das regiões de maior concentração visual.

Espera-se que a expansão do sistema para incluir a consulta preditiva da IA, sem a necessidade de novos testes com usuários reais, otimize o fluxo de trabalho dos designers, garantindo que a interface seja constantemente aprimorada com base em análises automatizadas. Espera-se, ainda, que o sistema auxilie os designers de UX tanto na análise preditiva quanto na precisão dos dados obtidos para o treinamento da IA. Cabe destacar que as novas tecnologias a serem incluídas futuramente têm o potencial de ser reconhecidas como fatores cruciais para a eficácia do projeto.