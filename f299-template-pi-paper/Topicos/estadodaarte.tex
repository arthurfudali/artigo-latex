Atualmente, a tecnologia é um fator chave quando se trata do processo criativo do design gráfico. A variedade de ferramentas a disposição do designer e sua eficiência são necessárias para atender de forma ideal as demandas do mercado. Nesta seção, como o software uEye se destaca no mercado de ferramentas e como ele se compara com outras tecnologias.


O primeiro estudo, conduzido por \textcite{GEORGES2016}, busca desenvolver uma ferramenta de avaliação para UX que utiliza rastreamento ocular e sinais psicológicos e comportamentais para criar heatmaps das interações do usuário com um sistema. Esses heatmaps não apenas mostram onde o usuário fixa o olhar durante a interação, mas também associam essas áreas aos estados emocionais do usuário.

Para a criação dos mapas de calor, foram empregados diversos processos, por exemplo, a triangulação de dados, que por sua vez foi realizada sincronizando as informações de rastreamento ocular com sinais fisiológicos, como frequência cardíaca e atividade elétrica da pele, o que possibilitou a obtenção de dados sobre o estado emocional do usuário. Além disso, um modelo de aprendizado de máquina foi utilizado para estimar a atividade cognitiva com base nos sinais fisiológicos, permitindo que os heatmaps não só identifiquem as áreas mais visualizadas, mas também indiquem aquelas que geram maior sobrecarga mental. 

Todos os dados gerados por esse processo passaram por uma etapa de normalização e colorização, na qual as áreas de interesse foram destacadas com cores proporcionais à intensidade dos estados psicológicos associados.

O experimento descrito no artigo envolveu 26 participantes, com a meta de analisar a carga cognitiva em relação à complexidade visual de interfaces de sites. Para o estudo, os pesquisadores selecionaram nove páginas iniciais de websites, distribuídas em três níveis de complexidade visual: baixo, médio e alto. Essas interfaces foram projetadas para induzir diferentes níveis de carga cognitiva nos usuários. As interfaces com maior complexidade visual exigiram mais recursos cognitivos, confirmando uma correlação direta entre a complexidade percebida e a carga cognitiva experimentada.

Os resultados mostraram que os heatmaps fisiológicos, correlacionaram-se significativamente com as avaliações subjetivas dos usuários em relação à complexidade visual das interfaces. Esses heatmaps se destacaram ao capturar com precisão as áreas com maior demanda cognitiva nas interfaces, superando as formas tradicionais de mapear o rastreamento ocular em termos de precisão na predição da carga cognitiva.
