A Organização das Nações Unidas (ONU) é uma instituição que visa estabelecer a paz, segurança e desenvolvimento global. A ONU conta com 193 países-membros que formam a Assembleia Geral, responsável por desenvolver as políticas da organização. Em 2015, como parte da Agenda 2030 para o Desenvolvimento Sustentável, foram criados 17 objetivos que abrangem desde a melhoria da indústria até o aprimoramento da saúde da população. Esse conjunto de metas constitui um plano de ação ambicioso para as pessoas, o planeta e a prosperidade. Este trabalho visa contribuir para o nono objetivo, que promove Inovação e Infraestrutura, ao desenvolver uma nova abordagem para analisar o design de interfaces de usuário.
\textcite{ODS2024}

As Interfaces de Usuário (UI) e a Experiência de Usuário (UX) são elementos fundamentais na interação entre um cliente e um produto digital. Essas duas áreas do design, quando combinadas, desempenham um papel crucial na satisfação do usuário com um sistema. A UI tem como foco a criação e melhoria dos aspectos visuais e interativos do produto. Seu objetivo é tornar a interface esteticamente agradável e funcional, facilitando uma interação mais eficiente e intuitiva entre o usuário e o produto.

Por outro lado, a UX abrange uma área maior dentro do Design, considerando todos os aspectos da interação do usuário com o produto, como a intuitividade do sistema e as emoções sentidas durante a utilização. Juntas, UI e UX trabalham em sinergia para criar uma experiência de uso satisfatória que atenda aos requisitos funcionais do usuário de forma simples e eficaz. 
\textcite{EBAC}

A experiência proporcionada pela UI é de extrema importância para o sucesso de um produto digital no mercado atual, altamente competitivo e centrado no usuário. Uma interface que seja simultaneamente funcional, esteticamente atraente e intuitiva de usar torna-se um fator decisivo na adoção inicial e na fidelização a longo prazo do usuário ao produto. Além disso, uma UI bem projetada pode significativamente reduzir a curva de aprendizado, minimizar erros do usuário, aumentar a eficiência nas tarefas e, consequentemente, elevar os níveis de satisfação e produtividade do usuário. Portanto, investir no desenvolvimento de uma UI de alta qualidade não é apenas uma questão de estética, mas uma estratégia fundamental para o sucesso e a sustentabilidade de qualquer produto digital no cenário tecnológico contemporâneo.

O teste da interface de usuário é frequentemente conduzido através de um procedimento conhecido como Teste de Usabilidade. Este método de avaliação consiste em uma entrevista estruturada, realizada em um ambiente controlado, onde há uma interação direta entre o pesquisador e o usuário de teste. Durante essa sessão, a atenção é dedicada principalmente às características específicas de uso e às funcionalidades específicas da interface em questão.

O objetivo deste tipo de teste é realizar uma análise do comportamento do usuário enquanto ele interage com o produto digital. Isso inclui observar como o usuário navega pela interface, quais elementos chamam sua atenção, onde encontra dificuldades e como resolve problemas. Além disso, o teste busca coletar informações sobre a eficiência com que o usuário realiza tarefas específicas, o tempo necessário para completá-las e o nível de satisfação geral com a experiência de uso.

Através deste processo, os pesquisadores podem identificar pontos fortes e fracos na interface, detectar possíveis obstáculos na experiência do usuário e coletar informações valiosas para futuras melhorias. Esta abordagem permite que as equipes de design e desenvolvimento refinem continuamente a interface, garantindo que ela atenda às necessidades do usuário e as expectativas relacionadas a qualidade de uso do sistema.

Essa análise é realizada considerando fatores como: o tempo que o usuário permanece em cada tela, o número de etapas necessárias para realizar uma tarefa, a quantidade de erros cometidos durante o processo e as sensações experimentadas ao usar o sistema. Tal avaliação detalhada só é possível quando um pesquisador observa atentamente cada passo do usuário. Por outro lado, numa análise não moderada — isto é, sem supervisão direta — o resultado pode ser comprometido, pois depende exclusivamente do feedback fornecido pelo usuário, e não de métricas objetivamente coletadas. Essa limitação afeta tanto a profundidade da análise quanto a credibilidade do teste.

\textcite{BELISIARIO2023, VIEIRA2019}


O eye tracking é uma técnica que consiste em usar o posicionamento dos olhos de uma pessoa para obter informações sobre onde ela está olhando. Isso pode ser feito usando luzes infravermelhas, que calculam exatamente onde a pessoa está olhando com base nas reflexões da luz na retina, ou por meio de câmeras que monitoram visualmente a posição dos olhos e identificam sua direção.

Quando aplicada a um teste de usabilidade, essa tecnologia pode fornecer informações valiosas ao pesquisador. Ela permite saber com precisão onde o usuário está olhando, o que ele procura e por quanto tempo olhou para cada elemento. Isso possibilita identificar problemas como textos confusos ou mal formatados, ou até mesmo elementos visuais que chamam atenção indevidamente e diferem da identidade visual do sistema, prejudicando sua usabilidade.

Um heatmap (ou mapa de calor) é uma representação gráfica de uma matriz de dados em que os valores são exibidos por meio de variações de cor para indicar diferentes intensidades. No contexto do eye tracking, os heatmaps são gerados com base nos pontos de atenção visual dos usuários em uma interface. Eles destacam as áreas que receberam maior tempo de atenção, seja por meio de fixações do olhar ou pela frequência com que foram visualizadas, utilizando cores intensas, geralmente em tons de vermelho. Por outro lado, as áreas menos visualizadas são representadas por cores frias, como tons de azul ou verde.

A inteligência artificial (IA) é uma área da ciência da computação dedicada ao estudo e desenvolvimento de sistemas capazes de realizar tarefas que normalmente requerem inteligência humana, como aprendizado, raciocínio, reconhecimento de padrões e interpretação. Esses sistemas utilizam algoritmos avançados para processar grandes volumes de informações, identificar padrões e, a partir disso, tomar decisões ou realizar previsões.

No contexto da análise de interfaces UX, a inteligência artificial (IA) pode ser aplicada para processar grandes volumes de dados provenientes de fontes como rastreamento ocular e avaliações dos usuários. Utilizando técnicas como aprendizado de máquina, a IA pode ser treinada a partir desses dados para reconhecer padrões de interação visual e identificar, de forma autônoma, possíveis melhorias na interface.

Assim, propomos por meio deste estudo a criação de um software (uEye) que usa as informações obtidas pelo rastreamento ocular de usuários durante o uso de telas de um sistema. Essas informações serão usadas para treinar uma inteligência artificial capaz de identificar, de forma rápida e precisa, possíveis divergências na Interface de Usuário de um sistema que poderiam prejudicar a Experiência do Usuário durante o uso. 


